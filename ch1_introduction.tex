\chapter{Introduction}

In recent years, due to an increase in the use of mobile phones and Internet around the World\cite[p. 131-133]{quinn1987impacts}, some developing nations are turning to a technology-based services industry with the hope that technology, and the Internet in particular, will serve as tools to speed up their economic growths towards sustainable development\cite{lall1998market}. However this has proven to be a difficult challenge for countries whose popular, and in most cases, official languages are not widely used elsewhere around the World. 

The reason why this is a big issue is because even though there is a lot of free and easily-accessible information on the Internet, around $77.9\%$ of the content on the web is in the top ten most popular languages (English, Chinese, Spanish, Arabic, Portuguese, Japanese, Malay, Russian, French, and German)\cite{internetworldstats}. Therefore without the use of a translator, the information on the Internet is inaccessible, and somewhat useless, to people who can't read or understand at least one of these popular languages. 

As an example, the Rwandan government, as part if its Vision 2020 plan, wanted to implement educational initiatives that would take advantage of free resources available on the Internet but most of these initiatives have been hindered by the fact that almost $90\%$ of Rwandans are only proficient in Kinyarwanda\cite{Samuelson2010} and there doesn't exist a competent translation service that can be used to translate to (or from) Kinyarwanda.  As a Rwandan, after realizing this, I felt inclined to do something, I decided to explore possible technical solutions that could be used as a starting point in order to deal with this particular problem of information scarcity caused by linguistic barriers.

My first step when embarking on this project was to search around and see if there is any other work that has been done in this domain and to find out if there are any already-developed tools that I could use to translate any of the popular language to Kinyarwanda or vice-versa. After failing to find any competent service that supported Kinyarwanda translation\footnote{Kinyarwanda is one the langauges currently in development at Google Translate, but it is not clear whether it will be one of the supported languages anytime soon\cite{googletranslatecommunity}.}, I started looking for research being done in the domain of natural language translation, and that is how I found out about Statistical Machine Translation (SMT) systems. 

Before diving into the actual process of making a Kinyarwanda | English translator, I researched and tried out some of the freely-available SMT systems such as Moses (a complete SMT system), Cdec (a Python-based SMT decoder and aligner), and Joshua (a java-based SMT decoder). 

For building my model, I ended up using Moses for reasons discussed in my concluding remarks. In the process, I also learnt about the phrase-based model that is used heavily in the field of statistical machine translation\cite{koehn2003statistical}. In this paper, I will explain how that model works and how it is implemented and how it is used in the above-mentioned SMT systems, while focusing on Moses and its role in my project.